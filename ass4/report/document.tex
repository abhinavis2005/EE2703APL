\subsection{Assignment 4}\label{assignment-4}

\subsubsection{Approach followed}\label{approach-followed}

\begin{itemize}
\item
  I first create a keyboard layout from the qwerty\_layout in the file
  as in the keyboard heatmap programming quiz.
\item
  The keyboard layout is generated using matplotlib rectangles in thhe
  function \texttt{genKeyboardLayout}
\item
  My heatmap is a pixel grid of 58 * 16 pixels in x, y coordinates.
\item
  Takes in an input string from the user.
\item
  I then generate two arrays x, y each containing x, y coordinate of
  each key pressed for example:

  \begin{itemize}
  \tightlist
  \item
    if i press a which has coordinates (1.75, 2) 1.75 gets appended to
    x, and 2 gets appended to y
  \item
    This keeps track of the frequencies with which each coordinate in
    the keyboard has been pressed.
  \end{itemize}
\item
  I then call the \texttt{calculate\_key\_travel} function which
  calculates the total distance travelled for pressing the entire text.
  it follows the following logic -

  \begin{itemize}
  \tightlist
  \item
    for each character c in the \texttt{input\_string} it checks each
    key to be pressed, and adds the distance to each of those keys onto
    sum
  \item
    it does this byy going to the \texttt{characters} dictionary, which
    contains a tuple consisting of each key to be pressed for achieving
    a particular character.
  \item
    For eg, say we want to type A, the \texttt{characters} dictionary
    will have a key \texttt{A} with the element
    \texttt{(\textquotesingle{}Shift\_R\textquotesingle{},\ \textquotesingle{}a\textquotesingle{})},
    so it adds the distance for going to both shift\_R and a.
  \item
    returns sum
  \end{itemize}
\item
  if -a flag is passed using the command line, generates animation, else
  saves heatmap.png
\item
  -a flags creates animation, although it takes time for longer text
  samples.
\item
  -a calls the plot function (for animation) and otherwise calls the
  plot1 function (without animatoin)
\item
  The plot function works as follows

  \begin{enumerate}
  \def\labelenumi{\arabic{enumi}.}
  \tightlist
  \item
    Consists of heatmap array, containing an element for each pixel.
  \item
    Contains X, Y arrays which are 2d arrays constructed using
    \texttt{np.meshgrip}
  \item
    For each letter, i update corresponding values in the heatmap array,
    (inside a circle of radius 0.6 units) multiplied with a function
    that has decreasing values as it goes further away from the center
    of the key, to create a gradient effect.
  \item
    I define a custom colormap for the heatmap from blue to red and use
    \texttt{plt.imshow} function to ccreate the heatmap and save it to
    \texttt{heatmap.png}
  \item
    If i use the \texttt{plot} function with the -a flag for animation,
    i create a function update, which gradually increments the heatmap
    array, updates the artist , used as an arguement to
    \texttt{FuncAnimation} function from numpy, and saves it to
    \texttt{animation.gif}
  \end{enumerate}
\end{itemize}

\subsubsection{The QWERTY LAYOUT FORMAT}\label{the-qwerty-layout-format}

\begin{itemize}
\tightlist
\item
  I follow the same layout format given in the programming quiz, the
  layout has two dictionaries :

  \begin{enumerate}
  \def\labelenumi{\arabic{enumi}.}
  \tightlist
  \item
    \texttt{keys} which contains a key for each key in the dictionary,
    and the value is a dictionary of the format
    \texttt{\{\textquotesingle{}pos\textquotesingle{}:\ (x,y),\ \textquotesingle{}start\textquotesingle{}:\textquotesingle{}home\_row\_key\textquotesingle{}\}}
    where the pos is the coordinate of the given key and \texttt{start}
    is the home row key to be used while typing that key
  \item
    \texttt{characters} which contains a character as key in the
    dictionary , and each character corresponds to a tuple of individual
    keys that have to be pressed for typing that key for example. to
    obtain A, we have to type both \texttt{Shift\_R} and \texttt{a}
  \end{enumerate}
\end{itemize}

\subsubsection{Results}\label{results}

\paragraph{Sample Text 1 :}\label{sample-text-1}

\begin{verbatim}
The environment is a vital part of our planet, providing the resources we need to live and thrive. It encompasses everything from forests and oceans to the air we breathe and the ecosystems that sustain biodiversity. Protecting the environment is crucial for maintaining the balance of nature and ensuring a healthy future for all living organisms. Human activities like pollution, deforestation, and climate change are threatening this delicate balance, making it imperative for us to take action. By adopting sustainable practices, conserving natural resources, and reducing our carbon footprint, we can preserve the environment for future generations.
\end{verbatim}

Distance travelled for qwerty layout = 483.68 units

Heatmap

\begin{figure}
\centering
\includegraphics{./images/image1.png}
\caption{Heatmap}
\end{figure}

Distance travelled for dvorak layout = 295.28 units

Heatmap

\begin{figure}
\centering
\includegraphics{./images/image2.png}
\caption{Heatmap}
\end{figure}

Distance travelled for colemak layout = 208.830 units

Heatmap

\begin{figure}
\centering
\includegraphics{./images/image3.png}
\caption{Heatmap}
\end{figure}

\paragraph{Sample text 2 :}\label{sample-text-2}

\begin{verbatim}
Programming is the art of crafting solutions through code, transforming ideas into functional applications that drive modern technology. It involves writing instructions in various languages like Python, Java, or C++ to communicate with computers and automate tasks. Programmers solve complex problems by breaking them down into manageable steps, creating efficient and scalable systems. The process fosters logical thinking, creativity, and persistence as developers debug and optimize their code. From websites and apps to artificial intelligence and data processing, programming powers innovation and shapes the digital landscape, making it an essential skill in today's rapidly evolving world.
\end{verbatim}

Distance travelled for qwerty layout = 506.98 units

Heatmap

\begin{figure}
\centering
\includegraphics{./images/image4.png}
\caption{Heatmap}
\end{figure}

Distance travelled for dvorak layout = 373.016 units

Heatmap

\begin{figure}
\centering
\includegraphics{./images/image5.png}
\caption{Heatmap}
\end{figure}

Distance travelled for colemak layout = 269.971 units

Heatmap

\begin{figure}
\centering
\includegraphics{./images/image6.png}
\caption{Heatmap}
\end{figure}

\subsubsection{Steps to test code with different
layouts}\label{steps-to-test-code-with-different-layouts}

All the layout files have to be \textbf{imported} eg
\texttt{import\ qwerty\_layout} or \texttt{import\ dvorak\_layout}

all the layouts have been uploaded in the zip file.

in the \texttt{main} block, update the layout used in the line
\texttt{layout\ =\ qwerty\_layout}

\subsubsection{Notes}\label{notes}

\begin{itemize}
\tightlist
\item
  All the \texttt{layout.py} has to be in the same directory as the
  script.py, if default analysis keyboard needs to be changed, change
  the \texttt{layout} to the required \texttt{layout}
\end{itemize}
